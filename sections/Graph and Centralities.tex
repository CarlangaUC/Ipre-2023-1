%!TEX root = ../main/main.tex


\paragraph{Graphs} A graph is a pair $G = (V, E)$ where $V$ is a finite set and $E \subseteq V\times V$ is a finite relation. Each pair of vertices $(v_{1},v_{2})$ is called an edge and it is assumed undirected. A subgraph of $G$ is a graph $G'=(V', E')$ such that $V' \subset V$ and $E' \subset E$.

\paragraph{Path} A path in a graph is a sequence of non-repeated nodes connected through edges present in a graph. If there exists a path between two vertices, we say that they are connected. As an example, consider $x,y \in V$. A path can be a sequence $x,x_{1},x_{2},\cdots,y_{1},y_{2},\cdots,y_{n}$ with $x,x_{i},y_{i},y \in V$.


\paragraph{Centrality} The centrality of a graph is defined as an assignment of a number to nodes within a graph corresponding to their network position. The score of a centrality to a graph indicates the importance of the vertex in the graph. There exists a different number of centrality measures, so the importance of a vertex can varies from centrality measure to another. Also, a centrality measure has to be any function $C : VG \rightarrow R$ that assigns a score $C(v, G)$ to $v$ depending on its graph $G$. 

\paragraph{All-subgraphs centrality} Given a graph $G = (V, E)$ and a vertex $v \in V$, we denote by $A(v, G)$ the set of all connected subgraphs of $G$ that contain $v$, formally, $A(v, G) = \{S \subseteq G | v \in V (S)$ and S is connected \cite{RiverosS20}. Then all-subgraphs centrality of $v$ in $G$ is defined
as: 
$$
\text{AllSubgraphs}(v, G) = \log_{2}{|A(v, G)|}.
$$

\paragraph{Potential function} A potential function~\cite{RiverosSS23} is a function that measures the “potential” of every rooted tree, i.e.,
a tree with one node selected and the assessment depends on the selection. Now, a centrality measure admits some potential function if the comparison between two adjacent vertices is determined by the potential of their corresponding subtrees. Some centralities measures admits a potential function but it can not root trees, but, all centralities that root trees has to have a potential function.

\paragraph{Trees} A tree is an undirected graph in which any two vertices are connected by only and only one path. A rooted tree is refereed as a tree with a vertex who serves as the "root" of the tree, being a references to the others  vertices in the tree. A neighborhood of a vertex v is denoted as $N_{G}(v)$, and contains all vertex (u) if ${u,v} \in V$ then $u \in N_{G}(v)$.

\paragraph{Previous Results} In a previous paper~\cite{RiverosSS23}, people find different properties that a potential function should have. A potential function $f$ is
symmetric over trees if, and only if, for every tree $T$ and every pair of adjacent vertices $u, v \in V(T)$, it holds that $f(u, T_{u,v}) < f(v, T)$. Also, defining this potential function implies using a monoid in the way that: Let $(M, \ast, 1)$ be a monoid and $l : M \rightarrow M$ a leaf-function. The potential function $f_{\ast,l}$ consistently roots trees whenever (1) $x < l(x)$ for every $x$, $(2)$ $l$ is monotonic, and (3) $(M, \ast, 1)$ is positively ordered~\cite{RiverosSS23}.


