%!TEX root = ../main/main.tex


In mathematics, the Strahler number or Horton-Strahler number of a tree is a numerical measure of its branching complexity. This measure has appeared in different contexts, such as animal respiratory systems \cite{horsfield1976some} and random trees \cite{devroye1996horton} (see also~\cite{EsparzaLS14} for more applications). These different applications of this concept raise the question of whether there exists a relation between the Strahler number and a centrality measure. For this purpose, we will use the tools we defined previously and furthermore provide mathematical proofs.

One may assign a Strahler number to all nodes of a tree, in bottom-up order, as follows:
\begin{enumerate}
\item 
If the node is a leaf (has no children), its Strahler number is one.
\item
If the node has one child with Strahler number i, and all other children have Strahler numbers less than i, then the Strahler number of the node is i again.
\item
If the node has two or more children with Strahler number i, and no children with greater number, then the Strahler number of the node is i + 1.
\end{enumerate}


