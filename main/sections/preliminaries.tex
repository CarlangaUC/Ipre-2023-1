%!TEX root = ../main/main.tex

%!TEX root = ../main/main.tex

\paragraph{Graphs} A graph is a pair $G = (V, E)$ where $V$ is a finite set and $E \subseteq V\times V$ is a finite relation. Each pair of vertex let be $(v_{1},v_{2})$ , $v_{1},v_{2} \in V$ this set of vertex its called edges and can be directed or not, so the edges $\in E$. A path in a graph is a sequence of non-repeated nodes connected through edges present in a graph, if there exist a path between two vertex, this vertex are called connected. An example, let be $x,y \in V$ a path can be expressed so $\{x,x_{1},x_{2},\cdots,y_{1},y_{2},\cdots,y_{n} \}$ with $x,x_{i},y_{i},y \in V$

\paragraph{Trees} A tree is an undirected graph in which any two vertices are connected by only and only one path. A rooted tree is refereed as a tree with a vertex who serves as the "root" of the tree, being a references to the others  vertices in the tree. A neighborhood of a vertex v is denoted as $N_{G}(v)$, and contains all vertex (u) if ${u,v} \in V$ then $u \in N_{G}(v)$.

\paragraph{Centrality} The centrality of a graph is defined as a assignation of a number to nodes within a graph corresponding to their network position, the application of a centrality to a graph indicates the importance of the vertex in the graph. There exist a different number of centrality measures, so the importance of a vertex changes from centrality measure to another, also, a centrality measure has to be any function $C : VG \rightarrow R$ that assigns a score $C(v, G)$ to $v$ depending on its graph $G$. An example of a centrality measure would be the one who was previously defined in the paper associated, such as, F-subgraph
centrality (denoted by $C_{F} (v, G))$:
\begin{equation}
    C_{F} (v, G) := \log (|F(v, G)|)
\end{equation}

\cristian{Agrega estas referencias \cite{RiverosS20} y \cite{RiverosSS23}. También agrega esta sobre Strahler number~\cite{EsparzaLS14} y esta otra \cite{stahler1952hypsometric}} 

\paragraph{Strahler number} In mathematics, the Strahler number or Horton–Strahler number of a mathematical tree is a numerical measure of its branching complexity. One may assign a Strahler number to all nodes of a tree, in bottom-up order, as follows:\\
If the node is a leaf (has no children), its Strahler number is one.\\
If the node has one child with Strahler number i, and all other children have Strahler numbers less than i, then the Strahler number of the node is i again.\\
If the node has two or more children with Strahler number i, and no children with greater number, then the Strahler number of the node is i + 1.


